\documentclass[11pt,a4paper]{book}
\usepackage[utf8]{inputenc}
\usepackage[spanish]{babel}
\usepackage{amsmath,amssymb}
\usepackage[margin=3cm]{geometry}
\usepackage{biblatex}
\addbibresource{S0022247X97951945.bib}
\title{CÁLCULO INTEGRAL}
\author{Luis Roca \\ 
Universidad Nacional de Ingeniería}
\date{2019}
\begin{document}
\frontmatter
\maketitle
\tableofcontents
\mainmatter
\chapter{Integral Indefinida}
\section*{Introducción}
\section{Antiderivadas}
Decimos que $F$ es una antiderivada de $f$ si $F^\prime=f$\footnote{Antiderivada o primitiva}.

\framebox[\textwidth][l]{
\parbox{\textwidth}{Por ejemplo $\cos$ es una primitiva de $-\sen$. Por ejemplo $\cos$ es una primitiva de $-\sen$.  

Por ejemplo $\cos$ es una primitiva de $-\sen$.  Por ejemplo $\cos$ es una primitiva de $-\sen$.}}
\section{Integración por partes}
\section{Cambio de variable}
\section{Sustitución trigonométrica}
La demostración se encuentra en \cite{GANGULY1997315}
\section{Derivadas}
\fbox{\parbox{\textwidth}{
En el instante $t$, la derivada $f^\prime$ o $df/dt$ es:
\begin{equation}
    f^\prime(t)=\lim_{\Delta t\to 0} \frac{f(t+\Delta t)-f(t)}{\Delta t}
    \label{eqn:01}
\end{equation}
}}

Veamos la ecuación (\ref{eqn:01})
\printbibliography
\end{document}
