\documentclass[a4paper,12pt]{article}
\usepackage[utf8]{inputenc}
\usepackage{amsmath}
\usepackage[spanish]{babel}
\title{Programación  y texto científico}
\author{Luis Roca}
\date{Marzo 2019}
\begin{document}
\maketitle
\section{Ejercicios}
\begin{enumerate}
    \item Calcule la siguiente integral:
\begin{equation*}
    I=\int \frac{2\ln x + x\sec^2 x}{
    x\sqrt{\ln^2 x+\tan x}} dx
\end{equation*}
Solución:

Hacemos el cambio $u=\ln^2 x+\tan x$ entonces $xdu=2\ln x+x\sec^2 x dx$, luego
\begin{equation*}
   I=\int \frac{1}{\sqrt{u}}du=2\sqrt{u}+C=2\sqrt{\ln^2 x+\tan x}+C
\end{equation*}
\item Calcule la siguiente integral:
\begin{equation*}
    I=\int \frac{\arcsen (x)}{5x^2\sqrt{1-x^2}} dx
\end{equation*}
\item Calcule mediante sumas de Riemann, \Big(sugerencia: $\sum_{j=1}^{n} a_j=a\frac{1-a^n}{1-a}$ \Big)
\begin{equation*}
    I=\int_{-3}^{-1} [ (x+2)^2+e^{-x} ] dx
\end{equation*}
\end{enumerate}
\end{document}
