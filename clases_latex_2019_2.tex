\documentclass[a4paper,12pt]{report}
\usepackage[utf8]{inputenc}
\usepackage{amsmath}
\usepackage[margin=2cm]{geometry}
\usepackage[spanish]{babel}
\usepackage{graphicx}
\usepackage{pgf,tikz,pgfplots}
\pgfplotsset{compat=1.15}
\usetikzlibrary{babel}
\usepackage{mathrsfs}
\usetikzlibrary{arrows}

\begin{document}
\chapter{Clase Viernes 15 de noviembre}
\begin{center}
 \Large
    \underline{EXAMEN DE MEDIO CURSO}
\end{center}
\begin{enumerate}
    \item Calcule la siguiente integral:\hfill (3 puntos)
\[   
\int \frac{2\ln x + x\sec^2 x}{
    x\sqrt{\ln^2 x+\tan x}} \,dx
\]
\item Calcule la siguiente integral:\hfill (3 puntos)
\[   
\int \frac{{\rm sgn}x\arcsen x}{
    5x^2\sqrt{1-x^2}} \,dx
\]
\item Calcule mediante sumas de Riemann 
(sugerencia: $\sum_{j=1}^n a_j=\frac{a(1-a^n)}{1-a}$ ):\hfill (3 puntos)
\[   
\int_{-3}^{-1} [ (x+2)^2+e^{-x}] \,dx
\]
\item Sea $f(x)=\int_0^x e^{\sen t}\, dt$. Si
$\int_0^\pi e^{\sen t}\, dt\approx 6,21$; calcule:
\hfill (3 puntos)
\[
\int_0^\pi f(x)\sen x \, dx
\]
\end{enumerate}
\chapter{Clase jueves 21 de noviembre}
\section{Figuras}

\definecolor{dtsfsf}{rgb}{0.82,0.18,0.18}
\definecolor{rvwvcq}{rgb}{0.08,0.39,0.75}
\definecolor{sexdts}{rgb}{0.18,0.49,0.19}
\begin{figure}[h]
    \centering
\begin{tikzpicture}[line cap=round,line join=rect,>=triangle 45,x=1cm,y=1cm]
\begin{axis}[
x=1cm,y=1cm,
axis lines=middle,
%ymajorgrids=true,
%xmajorgrids=true,
xmin=-4,xmax=5,ymin=-3,ymax=5,
xtick={-7,-6,...,9},
ytick={-9,-8,...,7},]
\clip(-7.17,-9.49) rectangle (9.49,7.13);
\draw [line width=2pt,color=red,domain=-4:5] plot(\x,{(-0--2*\x)/1});
\draw [line width=2pt,color=blue,domain=-4:5] plot(\x,{(--2-1*\x)/1});
\draw [line width=2pt,color=cyan,domain=-4:5,samples=100] plot(\x,{2*sin(pi*deg(\x))});
\draw [line width=2pt,color=green] (-2,-3) -- (-2,4);
\begin{scriptsize}
\draw[color=sexdts] (3,3.5) node {\Large $y=2x$};
\draw[color=rvwvcq] (3,0.5) node {\Large $y=2-x$};
\draw[color=dtsfsf] (-1.5,0.5) node {\Large $x=-2$};
\end{scriptsize}
\end{axis}
\end{tikzpicture}    
    \caption{Caption}
    \label{fig:my_label}
\end{figure}

\begin{figure}[h]
    \centering
\begin{tikzpicture}[line cap=round,line join=rect,>=triangle 45,x=1cm,y=1cm]
\begin{axis}[
x=1cm,y=1cm,
axis lines=middle,
%ymajorgrids=true,
%xmajorgrids=true,
xmin=-4,xmax=5,ymin=-3,ymax=5,
xtick={-7,-6,...,9},
ytick={-9,-8,...,7},]
\clip(-7.17,-9.49) rectangle (9.49,7.13);
\draw[line width=2pt,color=black, smooth,samples=100,domain=0:6.283185307179586] plot[parametric] function{3*cos((5*t))*cos((t)),3*cos((5*t))*sin((t))};
\begin{scriptsize}
\draw[color=brown] (3,3.5) node {\Large $r=3\sen(5\theta)$};
\end{scriptsize}
\end{axis}
\end{tikzpicture}    
    \caption{Caption}
    \label{fig:my_label}
\end{figure}





\begin{figure}[h]
\centering
\includegraphics[width=10cm]{CNX_Precalc_Figure_08_04_016new2.jpg}
\caption{Curvas polares}\label{fig:polares}
\end{figure}
\section{Tablas}
\begin{table}[h]
    \centering
\begin{tabular}{||c|c||}
\hline
 UNO &  DOS  \\
\hline
 1 &  a \\
 2 &  b \\   \hline     
\end{tabular}
\caption{Datos observados}\label{tabla:datos}
\end{table}

Vemos en el cuadro \ref{tabla:datos} y en la figura \ref{fig:polares}.

\listoffigures
 
\listoftables

\end{document}

