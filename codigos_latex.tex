\documentclass[12pt,a4paper]{article}
\usepackage[spanish]{babel}
\decimalpoint
\usepackage{indentfirst}
\usepackage[margin=1.5cm]{geometry}
\usepackage{amssymb}
\usepackage{amsfonts}
\usepackage{amsmath}
\usepackage[utf8]{inputenc}
\usepackage{graphicx}
\usepackage{listings}
\usepackage{color}
\definecolor{gray97}{gray}{.97}
\definecolor{gray75}{gray}{.75}
\definecolor{gray45}{gray}{.45}
\lstset{% general command to set parameter(s)
 frame=Ltb,
 framerule=0pt,
 aboveskip=0.5cm,
 framextopmargin=3pt,
 framexbottommargin=3pt,
 framexleftmargin=0.4cm,
 framesep=0pt,
 rulesep=.4pt,
 backgroundcolor=\color{gray97},
 rulesepcolor=\color{black},
language=Scilab,
basicstyle=\small\ttfamily, % print whole listing small
keywordstyle=\color{blue}\bfseries\underbar,
% underlined bold black keywords
identifierstyle=, % nothing happens
commentstyle=\color{red}, % white comments
stringstyle=\ttfamily, % typewriter type for strings
showstringspaces=false,
numbers=left, 
numberstyle=\tiny, 
stepnumber=1, 
numbersep=5pt,
breaklines=false,
numberfirstline = true,
}
\usepackage{algorithm}
\usepackage{algpseudocode}
\floatname{algorithm}{Algoritmo}
\renewcommand{\algorithmicrequire}{\textbf{Entrada:}}
\renewcommand{\algorithmicend}{\textbf{fin}}
\renewcommand{\algorithmicif}{\textbf{si}}
\renewcommand{\algorithmicthen}{\textbf{entonces}}
\renewcommand{\algorithmicelse}{\textbf{sino}}
\renewcommand{\algorithmicwhile}{\textbf{mientras}}
\renewcommand{\algorithmicdo}{\textbf{hacer}}

\begin{document}
	\section{ALGORITMOS}

	\begin{algorithm}[ht]
%\captionsetup[algorithm]{name=Algoritmo}
		\caption{EL ALGORITMO DE NEWTON-RAPHSON}
		\begin{algorithmic}[1]
		    \State $i \gets  1.0$
			\While{$k\leq N$}
			\State $p \gets  p_0-\frac{f(p_0)}{f^\prime(p_0)}$
			\If{ $|p-p_0| < TOL$ }		
			\State SALIDA($p$)
			\EndIf
			\State $i \gets i+1$
			\State $p_0 \gets p$			
			\EndWhile
			\State SALIDA('El método fracasó después de $N$ iteraciones'); PARAR.
		\end{algorithmic}
	\end{algorithm}
	\section{PROGRAMAS}
	\begin{lstlisting}
R=input("Ingrese R=");
xk=1;
printf("%4s%8s\n","k","xk")
for(k=1:10)
 xk=xk*(2-R*xk);
 printf("%4d%8.3f\n",k,xk)
end
printf("\n 1/R=%f",xk)
\end{lstlisting}
\end{document}