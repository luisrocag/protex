\documentclass{article}
\usepackage[utf8]{inputenc}
\usepackage{amssymb}
\usepackage{amsmath}
\begin{document}
\begin{enumerate}
\item Para que valores de $a$ y $b$ el sistema 
\[
\begin{array}{ccccccc}
 3x & - & 2y & + &  z & = &  b  \\
 5x & - & 8y & + & 9z & = &  3  \\
 2x & + &  y & + & az & = & -1  \\
\end{array}
\]
tiene:
\begin{enumerate}
    \item Solución única.
    \item  No tiene solución
    \item Infinitas soluciones.
\end{enumerate}
\item Hallar una matriz $A\in \mathbb{R}^{3\times 3}$ no nula, de modo que $v_1 = (1, 0, -1)$, $v_2 = (0, 2, 1)$ y $v_3 = (3, 4, -1)$ sean soluciones del sistema homogéneo $Ax = 0$.
\item Demuestre que $\mathcal{L} = \{(x, y, z) \in  \mathbb{R}^3 | 2x + y - z = 0\}$ es un $\mathbb{R}$-espacio vectorial.
\item Probar que $\mathbb{R}^2$ es un $\mathbb{R}$-espacio vectorial con la suma $\oplus$ y el producto
definidos de la siguiente forma
\[ (x, y)\oplus
 (x^\prime , y^\prime ) = (x + x^\prime - 2, 
 y + y^\prime - 1) \]
 \[ r \circledast(x, y) = r(x - 2, y - 1) + (2, 1)\]
Este espacio se denotará $R^2_{(2,1)}$ para distinguirlo de $R^2$ con la suma y el producto usual. La notación se basa en que el (2, 1) resulta el neutro de la suma $\oplus$.
 \item Si la ecuación matricial
 $ A \begin{pmatrix}
 x \\ y
 \end{pmatrix}=\begin{pmatrix}
 b_1 \\ b_2 \end{pmatrix}
 $ tiene por solución $\begin{pmatrix}
 x \\ y
 \end{pmatrix}= \begin{pmatrix}
 3 \\ 0
 \end{pmatrix}+t\begin{pmatrix}
 -2 \\ 1
 \end{pmatrix}$
\end{enumerate}
\end{document}
